%%%%%%%%%%%%%%%%%%%%%%%%%%%%%%%%%%%%%%%%%
% Journal Article
% LaTeX Template
% Version 1.3 (9/9/13)
%
% This template has been downloaded from:
% http://www.LaTeXTemplates.com
%
% Original author:
% Frits Wenneker (http://www.howtotex.com)
%
% License:
% CC BY-NC-SA 3.0 (http://creativecommons.org/licenses/by-nc-sa/3.0/)
%
%%%%%%%%%%%%%%%%%%%%%%%%%%%%%%%%%%%%%%%%%

%----------------------------------------------------------------------------------------
%	PACKAGES AND OTHER DOCUMENT CONFIGURATIONS
%----------------------------------------------------------------------------------------
\documentclass[twoside]{article}

\usepackage{mathtools}
\usepackage{amsmath}

\usepackage{graphicx}
\graphicspath{ {figures/} }

\usepackage[usenames,dvipsnames]{color} % Required for specifying custom colors and referring to colors by name

\definecolor{MyRed}{rgb}{0.6, 0.0, 0.0} 
\definecolor{MyGreen}{rgb}{0.0,0.4,0.0} % Comment color
\definecolor{MyBlue}{rgb}{0.0, 0.0, 0.6}

\setlength\parindent{24pt}

\usepackage[pdftex]{hyperref} % For hyperlinks in the PDF
\hypersetup{
  colorlinks=true,
  linkcolor=MyBlue, 
  citecolor=MyRed,
  urlcolor= MyGreen
}

% make hyperlinks bold
\newcommand{\aref}[1]
 {\textbf{\autoref{#1}}}

\newcommand{\nref}[1]
 {\textbf{\nameref{#1}}}

\newcommand{\cc}[1]
 {\textbf{\cite{#1}}}

\usepackage{lipsum} % Package to generate dummy text throughout this template

\usepackage[sc]{mathpazo} % Use the Palatino font
\usepackage[T1]{fontenc} % Use 8-bit encoding that has 256 glyphs
\linespread{1.05} % Line spacing - Palatino needs more space between lines
\usepackage{microtype} % Slightly tweak font spacing for aesthetics

\usepackage[hmarginratio=1:1,top=32mm,columnsep=20pt]{geometry} % Document margins
\usepackage{multicol} % Used for the two-column layout of the document
\usepackage[hang, small,labelfont=bf,up,textfont=it,up]{caption} % Custom captions under/above floats in tables or figures
\usepackage{booktabs} % Horizontal rules in tables
\usepackage{float} % Required for tables and figures in the multi-column environment - they need to be placed in specific locations with the [H] (e.g. \begin{table}[H])
\usepackage{hyperref} % For hyperlinks in the PDF

\usepackage{lettrine} % The lettrine is the first enlarged letter at the beginning of the text
\usepackage{paralist} % Used for the compactitem environment which makes bullet points with less space between them

\usepackage{abstract} % Allows abstract customization
\renewcommand{\abstractnamefont}{\normalfont\bfseries} % Set the "Abstract" text to bold
\renewcommand{\abstracttextfont}{\normalfont\small\itshape} % Set the abstract itself to small italic text

\usepackage{titlesec} % Allows customization of titles
%\renewcommand\thesection{\Roman{section}} % Roman numerals for the sections
%\renewcommand\thesubsection{\Roman{subsection}} % Roman numerals for subsections
\titleformat{\section}[block]{\large\scshape\centering}{\thesection.}{1em}{} % Change the look of the section titles
\titleformat{\subsection}[block]{\large}{\thesubsection.}{1em}{} % Change the look of the section titles


%----------------------------------------------------------------------------------------
%	TITLE SECTION
%----------------------------------------------------------------------------------------

\title{\vspace{-15mm}\fontsize{24pt}{10pt}\selectfont\textbf{Article Title}} % Article title

\author{
\large
\textsc{Evan Miller}\\[2mm] 
\textsc{Kiran Vodrahalli}\\[2mm]
\textsc{Albert Lee}\\[2mm]
\vspace{-5mm}
}
\date{January 13, 2015}

%----------------------------------------------------------------------------------------

\begin{document}

\maketitle % Insert title

%----------------------------------------------------------------------------------------
%	ABSTRACT
%----------------------------------------------------------------------------------------

\begin{abstract}

\noindent \lipsum[1] % Dummy abstract text

\end{abstract}

%----------------------------------------------------------------------------------------
%	ARTICLE CONTENTS
%----------------------------------------------------------------------------------------

\begin{multicols}{2} % Two-column layout throughout the main article text

\section{Introduction} \label{sec:Intro}

\lipsum[1]

\subsection{Problem Statement} 

% introduction 

% give clear definition of problem and motivation here

% adapt proposal -> introduction stuff

\lipsum[2-3] % Dummy text

\subsection{Previous Work}

% describe past work in the areas here

\lipsum[2-3] % Dummy text

%------------------------------------------------

\section{Describing the Algorithm}

% to note: our goal was to make a real time query algorithm that USES
% the history (in condensed form) to help understand current time 
% we're remembering the past with less accuracy to save space (realistic model)
% in order to get good estimates of present trending
% this means we are not trying to query the past the way Hokusai is.

% how did we approach it 
\lipsum[1]

\subsection{Data Structures}
% -- we maintain datastructure in two parts: History, and Present
\lipsum[2-3] % Dummy text

\subsection{Algorithm Pseudocode} 

\lipsum[4] % Dummy text

%------------------------------------------------

\section{Analyzing the Algorithm}

\lipsum[1]

\subsection{Correctness}

\lipsum[2-3]

\subsection{Spatial Analysis}

\lipsum[2-3]

\subsection{Runtime Analysis}

\lipsum[2-3]

%------------------------------------------------

\section{Design Choices}

\lipsum[1]

%------------------------------------------------

\section{Results}


\lipsum[5] % Dummy text

\begin{equation}
\label{eq:emc}
e = mc^2
\end{equation}

\lipsum[6] % Dummy text

%------------------------------------------------

\section{Discussion}

\lipsum[7] % Dummy text

%------------------------------------------------
\section{Future Work} \label{sec:Future Work}

\lipsum[1]

%----------------------------------------------------------------------------------------
%	REFERENCE LIST
%----------------------------------------------------------------------------------------

\begin{thebibliography}{99} % Bibliography - this is intentionally simple in this template

\bibitem[Figueredo and Wolf, 2009]{Figueredo:2009dg}
Figueredo, A.~J. and Wolf, P. S.~A. (2009).
\newblock Assortative pairing and life history strategy - a cross-cultural
  study.
\newblock {\em Human Nature}, 20:317--330.
 
\end{thebibliography}

%----------------------------------------------------------------------------------------

\end{multicols}

\end{document}
